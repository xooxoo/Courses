\documentclass[]{article}
\usepackage{lmodern}
\usepackage{amssymb,amsmath}
\usepackage{ifxetex,ifluatex}
\usepackage{fixltx2e} % provides \textsubscript
\ifnum 0\ifxetex 1\fi\ifluatex 1\fi=0 % if pdftex
  \usepackage[T1]{fontenc}
  \usepackage[utf8]{inputenc}
\else % if luatex or xelatex
  \ifxetex
    \usepackage{mathspec}
  \else
    \usepackage{fontspec}
  \fi
  \defaultfontfeatures{Ligatures=TeX,Scale=MatchLowercase}
\fi
% use upquote if available, for straight quotes in verbatim environments
\IfFileExists{upquote.sty}{\usepackage{upquote}}{}
% use microtype if available
\IfFileExists{microtype.sty}{%
\usepackage{microtype}
\UseMicrotypeSet[protrusion]{basicmath} % disable protrusion for tt fonts
}{}
\usepackage[margin=1in]{geometry}
\usepackage{hyperref}
\hypersetup{unicode=true,
            pdftitle={Анализ обломков в районе Сатино, Калужская область},
            pdfauthor={Р``еографический факультет РњР``РЈ имени Рњ.Р'. Ломоносова},
            pdfborder={0 0 0},
            breaklinks=true}
\urlstyle{same}  % don't use monospace font for urls
\usepackage{graphicx,grffile}
\makeatletter
\def\maxwidth{\ifdim\Gin@nat@width>\linewidth\linewidth\else\Gin@nat@width\fi}
\def\maxheight{\ifdim\Gin@nat@height>\textheight\textheight\else\Gin@nat@height\fi}
\makeatother
% Scale images if necessary, so that they will not overflow the page
% margins by default, and it is still possible to overwrite the defaults
% using explicit options in \includegraphics[width, height, ...]{}
\setkeys{Gin}{width=\maxwidth,height=\maxheight,keepaspectratio}
\IfFileExists{parskip.sty}{%
\usepackage{parskip}
}{% else
\setlength{\parindent}{0pt}
\setlength{\parskip}{6pt plus 2pt minus 1pt}
}
\setlength{\emergencystretch}{3em}  % prevent overfull lines
\providecommand{\tightlist}{%
  \setlength{\itemsep}{0pt}\setlength{\parskip}{0pt}}
\setcounter{secnumdepth}{0}
% Redefines (sub)paragraphs to behave more like sections
\ifx\paragraph\undefined\else
\let\oldparagraph\paragraph
\renewcommand{\paragraph}[1]{\oldparagraph{#1}\mbox{}}
\fi
\ifx\subparagraph\undefined\else
\let\oldsubparagraph\subparagraph
\renewcommand{\subparagraph}[1]{\oldsubparagraph{#1}\mbox{}}
\fi

%%% Use protect on footnotes to avoid problems with footnotes in titles
\let\rmarkdownfootnote\footnote%
\def\footnote{\protect\rmarkdownfootnote}

%%% Change title format to be more compact
\usepackage{titling}

% Create subtitle command for use in maketitle
\newcommand{\subtitle}[1]{
  \posttitle{
    \begin{center}\large#1\end{center}
    }
}

\setlength{\droptitle}{-2em}

  \title{Анализ обломков в районе Сатино,
Калужская область}
    \pretitle{\vspace{\droptitle}\centering\huge}
  \posttitle{\par}
    \author{Р``еографический факультет РњР``РЈ имени
Рњ.Р'. Ломоносова}
    \preauthor{\centering\large\emph}
  \postauthor{\par}
      \predate{\centering\large\emph}
  \postdate{\par}
    \date{30 РЅРѕСЏР±СЂСЏ 2018 Рі}


\begin{document}
\maketitle

\subsection{ВВЕДЕНИЕ}

В данной работе рассматриваются результаты анализа обломков. Обломки
получены из зачисток в стенках оврагов в районе деревни Сатино,
Калужская область\footnote{Расположение деревни Сатино на
  \href{https://yandex.ru/maps/?clid=1985551-225\&ll=36.380161\%2C55.206673\&z=15}{Яндекс.Картах}}.
Всего было произведено 935 наблюдений в 7 оврагах за период 2017-2018
гг.

\subsection{МЕТОДИКА}

Для применения статистического анализа использовался язык
программирования R (версия 3.5.1)

\begin{verbatim}
## # A tibble: 7 x 5
##   Место     Количество Длина Ширина Высота
##   <chr>          <int> <dbl>  <dbl>  <dbl>
## 1 Волчий           168  3.61   2.54   1.59
## 2 Егоров           248  4.32   3.03   1.71
## 3 Излучина         100  4.49   3.28   1.99
## 4 Обцарский         94  5.02   2.70   1.28
## 5 Пойма            125  3.68   2.68   1.58
## 6 Рыжк_клад        100  4.03   2.72   1.64
## 7 Язвицы           100  4.61   3.27   1.88
\end{verbatim}

Производен дисперсионный анализ для различий в ширине обломков по
местонахождению:

\begin{verbatim}
##              Df Sum Sq Mean Sq F value  Pr(>F)    
## Place         6     67   11.23    9.95 1.1e-10 ***
## Residuals   928   1047    1.13                    
## ---
## Signif. codes:  0 '***' 0.001 '**' 0.01 '*' 0.05 '.' 0.1 ' ' 1
\end{verbatim}

Изучалась взаимосвязь между длиной и шириной обломка:

\begin{verbatim}
## 
## Call:
## lm(formula = B ~ A, data = rocks)
## 
## Residuals:
##    Min     1Q Median     3Q    Max 
## -6.898 -0.403 -0.016  0.454  2.289 
## 
## Coefficients:
##             Estimate Std. Error t value Pr(>|t|)    
## (Intercept)   0.8623     0.0690    12.5   <2e-16 ***
## A             0.4811     0.0153    31.4   <2e-16 ***
## ---
## Signif. codes:  0 '***' 0.001 '**' 0.01 '*' 0.05 '.' 0.1 ' ' 1
## 
## Residual standard error: 0.762 on 933 degrees of freedom
## Multiple R-squared:  0.514,  Adjusted R-squared:  0.513 
## F-statistic:  986 on 1 and 933 DF,  p-value: <2e-16
\end{verbatim}

\includegraphics{satino_files/figure-latex/unnamed-chunk-5-1.pdf}


\end{document}
